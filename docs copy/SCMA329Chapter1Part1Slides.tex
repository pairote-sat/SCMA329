% Options for packages loaded elsewhere
\PassOptionsToPackage{unicode}{hyperref}
\PassOptionsToPackage{hyphens}{url}
%
%\documentclass[
%]{book}
\documentclass[landscape, 20pt]{extreport}
\usepackage{amsmath,amssymb}
\usepackage{lmodern}
\usepackage{iftex}
\ifPDFTeX
  \usepackage[T1]{fontenc}
  \usepackage[utf8]{inputenc}
  \usepackage{textcomp} % provide euro and other symbols
\else % if luatex or xetex
  \usepackage{unicode-math}
  \defaultfontfeatures{Scale=MatchLowercase}
  \defaultfontfeatures[\rmfamily]{Ligatures=TeX,Scale=1}
\fi
% Use upquote if available, for straight quotes in verbatim environments
\IfFileExists{upquote.sty}{\usepackage{upquote}}{}
\IfFileExists{microtype.sty}{% use microtype if available
  \usepackage[]{microtype}
  \UseMicrotypeSet[protrusion]{basicmath} % disable protrusion for tt fonts
}{}
\makeatletter
\@ifundefined{KOMAClassName}{% if non-KOMA class
  \IfFileExists{parskip.sty}{%
    \usepackage{parskip}
  }{% else
    \setlength{\parindent}{0pt}
    \setlength{\parskip}{6pt plus 2pt minus 1pt}}
}{% if KOMA class
  \KOMAoptions{parskip=half}}
\makeatother
\usepackage{xcolor}
\IfFileExists{xurl.sty}{\usepackage{xurl}}{} % add URL line breaks if available
\IfFileExists{bookmark.sty}{\usepackage{bookmark}}{\usepackage{hyperref}}
\hypersetup{
  pdftitle={SCMA329 Practical Mathematical Financial Modeling},
  pdfauthor={Pairote Satiracoo},
  hidelinks,
  pdfcreator={LaTeX via pandoc}}
\urlstyle{same} % disable monospaced font for URLs
\usepackage[margin=1in]{geometry}
\usepackage{longtable,booktabs,array}
\usepackage{calc} % for calculating minipage widths
% Correct order of tables after \paragraph or \subparagraph
\usepackage{etoolbox}
\makeatletter
\patchcmd\longtable{\par}{\if@noskipsec\mbox{}\fi\par}{}{}
\makeatother
% Allow footnotes in longtable head/foot
\IfFileExists{footnotehyper.sty}{\usepackage{footnotehyper}}{\usepackage{footnote}}
\makesavenoteenv{longtable}
\usepackage{graphicx}
\makeatletter
\def\maxwidth{\ifdim\Gin@nat@width>\linewidth\linewidth\else\Gin@nat@width\fi}
\def\maxheight{\ifdim\Gin@nat@height>\textheight\textheight\else\Gin@nat@height\fi}
\makeatother
% Scale images if necessary, so that they will not overflow the page
% margins by default, and it is still possible to overwrite the defaults
% using explicit options in \includegraphics[width, height, ...]{}
\setkeys{Gin}{width=\maxwidth,height=\maxheight,keepaspectratio}
% Set default figure placement to htbp
\makeatletter
\def\fps@figure{htbp}
\makeatother
\setlength{\emergencystretch}{3em} % prevent overfull lines
\providecommand{\tightlist}{%
  \setlength{\itemsep}{0pt}\setlength{\parskip}{0pt}}
\setcounter{secnumdepth}{5}
\usepackage{booktabs}
\usepackage{amsthm}
\usepackage{LectureNoteMacro}
\usepackage{actuarialangle}
\usepackage{bbm}
\usepackage{mathtools}
\makeatletter
\def\thm@space@setup{%
  \thm@preskip=8pt plus 2pt minus 4pt
  \thm@postskip=\thm@preskip
}
\makeatother
\ifLuaTeX
  \usepackage{selnolig}  % disable illegal ligatures
\fi
\usepackage[]{natbib}
\bibliographystyle{apalike}

\title{SCMA329 Practical Mathematical Financial Modeling}
\author{Pairote Satiracoo}
\date{2022-08-14}

\usepackage{amsthm}
\newtheorem{theorem}{Theorem}[chapter]
\newtheorem{lemma}{Lemma}[chapter]
\newtheorem{corollary}{Corollary}[chapter]
\newtheorem{proposition}{Proposition}[chapter]
\newtheorem{conjecture}{Conjecture}[chapter]
\theoremstyle{definition}
\newtheorem{definition}{Definition}[chapter]
\theoremstyle{definition}
\newtheorem{example}{Example}[chapter]
\theoremstyle{definition}
\newtheorem{exercise}{Exercise}[chapter]
\theoremstyle{definition}
\newtheorem{hypothesis}{Hypothesis}[chapter]
\theoremstyle{remark}
\newtheorem*{remark}{Remark}
\newtheorem*{solution}{Solution}
\begin{document}
\maketitle

{
\setcounter{tocdepth}{1}
%\tableofcontents
}
\hypertarget{cashflows-interest-and-the-time-value-of-money}{%
\chapter{Cashflows, Interest and the Time Value of Money}\label{cashflows-interest-and-the-time-value-of-money}}

\hypertarget{introduction-to-financial-modelling}{%
\newpage \section{Introduction to Financial Modelling}\label{introduction-to-financial-modelling}}

A financial model is a financial representation of a real world
financial situation, which is either a mathematical or statistical model
that describes the relationship among the variables of the financial
problem. Here are some types of financial models.

\begin{itemize}
\item
  \textbf{Financial statement model:} The model includes three main
  components including income statement, cash flow statement and
  balance sheet. These are accounting reports issued by a company
  quarterly and annually that are used for decision making and
  performing financial analysis. (see
  \url{https://corporatefinanceinstitute.com/resources/knowledge/accounting/three-financial-statements/})
\item
  \textbf{Project finance models:} The model incorporates two main elements
  of the project including loans and debt repayment. It can be used to
  assess the risk-reward of lending to or investing in a long-term
  project, i.e.~it can be used to tell whether the project has enough
  cash to cover the debt in the long term. (see
  \url{https://www.wallstreetprep.com/knowledge/project-finance-model-structure/})
\item
  \textbf{Discounted cashflow model:} It is the model to value a company
  using the net present value of the business's future cashflows, or
  to estimate the value of an investment based on its future cash
  flows. (see
  \url{https://corporatefinanceinstitute.com/resources/templates/excel-modeling/dcf-model-template/})
\item
  \textbf{Pricing models:} This models the way prices are set within a
  market in order to maximise profits.
\end{itemize}

This chapter covers the basic concepts of calculating interest,
including simple and compound interest, the frequency of compounding,
the effective interest rate and the discount rate, and the present and
future values of a single payment.

\hypertarget{cashflows}{%
\newpage \section{Cashflows}\label{cashflows}}

Cashflows are amounts of money which are received (or income, positive
cashflows) or paid (or outgo, negative cashflows) at particular times.
Those payments arise from a financial transaction, e.g

\begin{itemize}
\item
  a bank account,
\item
  a loan,
\item
  an equity,
\item
  a zero-coupon bond: A bond is a fixed income instrument that
  represents a loan from an investor to a debtor either a government
  or a corporation. A zero-coupon bond is a bond that pays no interest
  during its life.
\item
  a fixed interest security: A fixed-income security is a debt
  instrument such as a bond or debenture that investors use to lend
  money to a company in exchange for interest payments.
\item
  an index-linked security: An index-linked bonds pay interest that is
  tied to an underlying index, such as the consumer price index (CPI).
  Index-linked bonds are issued by governments to mitigate the effects
  of inflation by paying a real return plus accrued inflation.
\item
  an annuity: An annuity is a series of payments made at regular
  intervals, such as equal monthly payments on a mortgage.
\item
  a capital project etc.
\end{itemize}

Cash received represents inflows, income or also called \textbf{positive
cashflows}, while money spent represents outflows, outgo or \textbf{negative
cashflows}. The net cashflow at a given point in time is the difference
between expenses and income.

\newpage \begin{example}
\protect\hypertarget{exm:unlabeled-div-1}{}\label{exm:unlabeled-div-1}

\emph{A series of payments into and out of a bank account is given as
follows:}

\begin{itemize}
\item
  \emph{payments into the account : ฿1000 on 1 January 2014 and ฿100 on 1
  January 2016}
\item
  \emph{payments out of the account : ฿200 on 1 July 2015, ฿300 on 1 July
  2016, and ฿400 on 1 January 2018}
\end{itemize}

\end{example}

In practice, cashflows can be represented by a timeline as can be
illustrated in this example.

\begin{figure}

{\centering \includegraphics{tikz-ex1-1} 

}

\caption{an example of a timeline}\label{fig:tikz-ex1}
\end{figure}

\hypertarget{interest-and-the-time-value-of-money}{%
\newpage \section{Interest and the Time Value of Money}\label{interest-and-the-time-value-of-money}}

This section introduces the time value of money using the concepts of
compound interest and discounting. The effect of interest rates on the
present value of future cash flows is discussed. The value of distant
cash flows in the present and current cash flows in the future are then
considered.

We illustrate the time value of money by considering the following
examples.

\newpage \begin{example}
\protect\hypertarget{exm:egpv}{}\label{exm:egpv}

\emph{An investor want to make a payment of ฿10000 in 2 years. Suppose that a
bank pays compound interest at 4\% per annum effective. How much should
the initial investment?}

\end{example}

\textbf{Note} The amount we need to invest now (i.e.~the initial investment
in this example) is called the \emph{present value (PV)} or \emph{discounted
value} of the payments.

\textbf{Solution:} The interest for year 1 is

\[ X \cdot 0.04.\] For year 2 the principal is

\[ X + X \cdot 0.04 = X \cdot (1 + 0.04)\] so that the interest for the
year is

\[ X \cdot (1 + 0.04) \cdot 0.04.\]

By the end of 2 years an initial payment of ฿X will have accumulated to:

\[X\cdot (1 + 0.04) + X \cdot (1 + 0.04) \cdot 0.04 =   X  \cdot  1.04^2 = 10000.\]
Hence,

\[X = \frac{10000}{1.04^2} = 9245.56213,\]

\textbf{Note} We refer to the amount to which the capital accumulates with
the addition of interest as \emph{accumulation} or \emph{accumulated value}.

\newpage \begin{example}
\protect\hypertarget{exm:unlabeled-div-2}{}\label{exm:unlabeled-div-2}

\emph{Consider the following arguments}

\begin{itemize}
\item
  \emph{It is obvious that you would prefer to have ฿1100 now than ฿1000
  now.}
\item
  \emph{If we receive and hold ฿1 now, then it is worth more than receiving
  and holding ฿1 at some time in the future? Why is this?}
\item
  \emph{Is it obvious that your would be better off with ฿1100 in 2 years
  than ฿1000 now?}
\end{itemize}

\end{example}

\textbf{Solution:}

For the second argument, this one baht will grow to \(1 + r\) in the first
year, \((1 + r)^2\) in two years, and so on. These amounts are clearly
worth more than receiving and holding ฿1 at the same time in the future.

For the last argument, we need to compare the values of the amounts
**received at different times. To do this, we can look at the today's
values of ฿1100 received in 2 years assuming that we can invest at an
annual interest rate of \(r\) percent.

The present value of this amount \(X\) in year 2 is \[ 1100/(1 + r)^2.\] Assuming \(r = 5\%\), the present value of \(X\) is 997.7324263.

Comparing the values in today's baths, it is better to
have ฿1000 now than to have ฿1100 in 2 years.

\textbf{Notes} From the above example,

\begin{enumerate}
\def\labelenumi{\arabic{enumi}.}
\item
  One can deposit or invest ฿1 now and will receive ฿1 back and a
  reward called \emph{interest} at some point in the future. Because of its
  potential earning power, money in the present is worth more than an
  equal amount in the future. This is a fundamental financial
  principle known as \textbf{the time value of money}.
\item
  At a given point of time, cash has a monetary value, but also has a
  \emph{time value}.
\item
  The amount deposited or invested is called \emph{capital} or \emph{principal}.
\end{enumerate}

\hypertarget{simple-interest}{%
\newpage \subsection{Simple interest}\label{simple-interest}}

Simple interest is a calculation of interest that does not take into
account the effect of compounding. Under simple interest, the amount of interest that accrues over time is proportional to the length of the period.

Suppose an amount \(C\) is deposited in
an account that pays simple interest at the rate of \(i\)\% per annum. Then
after \(n\) years the deposit will have accumulated to
\[C( 1 + i \cdot n).\] Hence, the interest accrued over \(n\) years is
\[\text{Simple Interest}  = C \cdot i \cdot n.\]

\textbf{Note} Auto loans and short-term personal loans are usually simple
interest loans.

\newpage \begin{example}
\protect\hypertarget{exm:unlabeled-div-3}{}\label{exm:unlabeled-div-3}

\emph{An investor deposits ฿10000 in a bank account that pays simple interest
at a rate of 5\% per annum. Calculate}

\begin{enumerate}
\def\labelenumi{\arabic{enumi}.}
\item
  \emph{interest he will earn after the first two years.}
\item
  \emph{interest he will earn after the first three months.}
\end{enumerate}

\end{example}

\textbf{Note} When \(n\) is not an integer, interest is paid on a pro-rate
basis (in proportion).

\textbf{Solution:}

\begin{enumerate}
\def\labelenumi{\arabic{enumi}.}
\item
  At the end of 2 years the interest earned is
  \[10000 \cdot 0.05 \cdot 2 = 1000.\]
\item
  At the end of 3 months the interest earned is
  \[10000 \cdot 0.05 \cdot \frac{3}{12} = 125.\] Alternatively, the
  interest per month is 5\%/12 = 0.4167\% and hence the interest earned
  can be calculated as \[10000 \cdot 0.004167 \cdot 3 = 125.\]
\end{enumerate}

\hypertarget{compound-interest}{%
\newpage \subsection{Compound interest}\label{compound-interest}}

In compound interest, the accumulated amount over a period of time is the capital of the following period. Therefore, a capital of 1 unit at the end of the year increases to \(1 + i\) units, which becomes the capital for the following year.

For year 2, the principal is \(1 + i\) and the interest for the year is
\(( 1 + i ) \cdot i\). By the end of 2 years, an initial payment of 1 will have accumuulated to
\[ (1+i) + ( 1 + i ) \cdot i = (1+i)^2.\]

As this progression continues, the accumulated amount of \(X\) units at the end of year \(n\) becomes
\[ X\cdot(1 + i)^n. \]

\theNote In this case, we can take money out and reinvest it as new capital illustrated in the timeline.

\begin{figure}

{\centering \includegraphics{tikz-ex2-1} 

}

\caption{a timeline of compounding interest}\label{fig:tikz-ex2}
\end{figure}

\begin{exercise}
\protect\hypertarget{exr:unlabeled-div-4}{}\label{exr:unlabeled-div-4}

(Excel) Use Excel to create a table showing the accumulated amounts at the end of each year for 15 years for a principal of ฿100 under the simple interest approach and the compound interest approach with \(r = 6\%\) for both cases.
Discuss the results obtained (How long does it take to double the investment? How much will the principal grow over a 15-year period?)

\end{exercise}

The effect of compounding is to increase the total amount of accumulation. The effect is greater when the interest rate is high. This example shows two examples of the accumulated amount of ฿100 under the simple interest approach and the compound interest approach. As can be seen, the compound interest method makes the principle increase much faster than the simple interest method when the interest rate is high.

\hypertarget{frequency-of-compounding}{%
\newpage \section{Frequency of Compounding}\label{frequency-of-compounding}}

Even though the interest rate is typically expressed in annual terms, an investment's interest is frequently paid more frequently than once per year. For example, a savings account may offer an interest rate of 4\% per year, credited quarterly. This interest rate is usually referred to as \textbf{nominal rate of interest}, i.e., 4\% due four times per year.

We will see that the frequency of interest payments, also known as the frequency of compounding, has a significant impact on the total amount accrued and the interest collected. Consequently, it is crucial to precisely specify the rate of interest.

We use \(i^{(m)}\) to represent the nominal rate of interest payable \(m\) times a year in order to underline the significance of the frequency of compounding. Therefore, \(m\) is the
frequency of compounding per year and \(1/m\) year is the \textbf{compounding period} or \textbf{conversion period}.

\textbf{Note} The nominal rate of interest payable \(m\) times per period is also known as the rate of interest convertible \(m\)thly or compounded \(m\)thly.

\newpage \begin{example}
\protect\hypertarget{exm:unlabeled-div-5}{}\label{exm:unlabeled-div-5}

Calculate the accumulated value in 1 year of a deposit of ฿100 in a saving account that earns interest at 10\% payable quarterly.

\end{example}

\textbf{Solution:} In this example, the nominal rate of interest of \(i^{(4)} = 10\%\) p.a. convertible
quarterly means an interest rate of 10\%/4 = 2.5\% per quarter. In this case, the interest rate of 2.5\% is called \emph{effective interest}. The effective interest rate of \(i\) per unit of time (which may be month, quarter, etc.) is the amount of interest received at the end of a unit of time per ฿1 invested at the beginning of that unit.

Therefore, the nominal interest rate \(i^{(4)} = 10\%\) is equivalent to an
\emph{effective interest rate} of \(2.5\%\) per quarter.
The accumulated value in
1 year is \(100 (1 + \frac{10\%}{4})^4 = 100 (1 + 2.5\%)^4 = 110.3813\).

Note that after compound interest is
taken into account, the interest income of an investor at the quarterly
convertible nominal interest rate of 10\% p.a. is 10.3813 (or 10.3813\%.p.a. effective)

\newpage \begin{example}
\protect\hypertarget{exm:unlabeled-div-6}{}\label{exm:unlabeled-div-6}

\emph{At a rate of 12\% p.a. effective, draw a timeline to show cashflows if
฿100 is invested at the start of the year.}

\end{example}

\textbf{Solution:} The accumulated value of ฿100 at the end of the year is
\(100 (1 + 12\%) = 112\).

\newpage \begin{example}
\protect\hypertarget{exm:unlabeled-div-7}{}\label{exm:unlabeled-div-7}

\emph{At a rate of 12\% p.a. compounding quarterly, draw a time line to show
cashflows if ฿100 is invested at the start of the year.}

\end{example}

\textbf{Solution:} The nominal interest rate \(i^{(4)} = 12\%\) is equivalent
to an effective interest rate of \(3\%\) per quarter. The accumulated
value in 1 year is \(100 (1 + 3\%)^4 = 112.55\). After compound interest
is taken into account, the interest income of an investor at the
quarterly convertible nominal interest rate of 10\% p.a. is 12.55 (or
12.55\%. p.a. effective)

\textbf{Note} \(i^{(m)}\) is a nominal rate of interest which is equivalent to \(i^{(m)}/m\) applied for
each \(m\)th of a period. The interest is paid \(m\) times per measurement period.

The value at time \(n\) can be considered as the \textbf{annuity} with a cashflow of \(i^{(m)}/m\) per period for \(n\) years together with the capital at time \(n\) as shown in the following figure.
Therefore, the accumulated value in 1 year can also be calculated as \(100( 1 + 0.03 s_{\angl{4}}^{3\%})\). The concept of annuity will be discussed in the subsequent section.

\begin{figure}

{\centering \includegraphics{tikz-ex5-1} 

}

\caption{Frequency of Compouding vs Annuity}\label{fig:tikz-ex5}
\end{figure}

In general, we have

\begin{figure}

{\centering \includegraphics{tikz-ex4-1} 

}

\caption{Frequency of Compouding vs Annuity}\label{fig:tikz-ex4}
\end{figure}

\hypertarget{effective-rate-of-interest}{%
\newpage \subsection{Effective rate of interest}\label{effective-rate-of-interest}}

The compounding frequency affects the accumulated amount. As a result, it may be inaccurate to compare two investment strategies only based on their nominal rates of return without also taking into account their frequency of compounding. It is necessary to compare different investment strategies on an equal basis. The measure known as the \textbf{effective interest rate} is often used for this purpose.

The effective rate of interest of \(i\) per time unit is the amount of
interest received at the end of one time unit per ฿1 invested at the
start of that time unit.

\newpage \begin{example}
\protect\hypertarget{exm:unlabeled-div-8}{}\label{exm:unlabeled-div-8}

\emph{An investor invests ฿1 at 7.5\% p.a. (per annum) effective. Then}
\(i = 0.075\). Calculate the value of investment after one year.

\end{example}

\textbf{Solution:} The value of investment after one year at this rate is
\[1 \times ( 1 + 0.075) = 1.075.\]
In particular, the amount of
interest received at the end of the year per ฿1 invested is 0.075.

\newpage \begin{example}
\protect\hypertarget{exm:unlabeled-div-9}{}\label{exm:unlabeled-div-9}

\emph{An investor invests ฿1000 at 5.25\% per half-year effective. Then}
\(i = 0.0525\). Calculate the value of investment after half a year.

\end{example}

\textbf{Solution:} The value of investment after half year at this rate is
\[1000 \times ( 1 + 0.0525) = 1052.5.\]
Again, the amount ofinterest received at the end of the quarter of ฿1 invested is 0.0525.

\textbf{Note} The time unit is an \textbf{essential part of the definition}.

\newpage \begin{example}
\protect\hypertarget{exm:unlabeled-div-10}{}\label{exm:unlabeled-div-10}

\emph{An investor invests ฿1 at effective rate} \(i\)\% per time unit for \(n\)
time units. Calculate the value of investment after two, three,
\(\ldots\), \(n\) time units.

\end{example}

\textbf{Note} Here we assume that we can take money out and reinvest it as
new capital (see the timeline).

\begin{center}\includegraphics{tikz-ex6-1} \end{center}

\newpage \begin{example}
\protect\hypertarget{exm:unlabeled-div-11}{}\label{exm:unlabeled-div-11}

\emph{An investor invests ฿200 at} \(3\)\% pa effective. What will be the
deposit have accumulated to after 5 years.

\end{example}

\textbf{Solution:} The deposit accumulates to
\(200 \cdot (1.03)^5 = 231.854815\) after 5 years.

\newpage \begin{example}
\protect\hypertarget{exm:unlabeled-div-12}{}\label{exm:unlabeled-div-12}

Consider the following problems.

\begin{enumerate}
\def\labelenumi{\arabic{enumi}.}
\item
  \emph{An investor invests ฿500 at} \(2.75\)\% per quarter effective. What
  will be the deposit have accumulated to after 9 months.
\item
  \emph{An investor invests ฿2000 at} \(6\)\% per half-year effective. What
  will be the deposit have accumulated to after 2 years.
\end{enumerate}

\end{example}

\textbf{Solution:}

\begin{enumerate}
\def\labelenumi{\arabic{enumi}.}
\item
  Accumulating the 500 for 9 months at this rate gives
  \[500 \cdot (1.0275)^3 = 542.394773.\]
\item
  After 2 years the accumulation is
  \[2000 \cdot (1.06)^4 = 2524.95392.\]
\end{enumerate}

\textbf{Notes}

\begin{enumerate}
\def\labelenumi{\arabic{enumi}.}
\item
  The model under the effective rate of interest condition is a
  model of \emph{compound interest}, where interest is earned on interest
  previously earned. Unless state otherwise, we shall assume that \(i\) is
  the compound interest rate.
\item
  In practice, it is easier to work with the effective rate of
  interest which is defined in a suitable time unit.
\end{enumerate}

The following formula can be used to convert between the effective rate
\(i\) p.a. and the nominal rate \(i^{(m)}\) p.a.:
\[( 1 + i) = \left( 1 + \frac{i^{(m)}}{m}\right)^m.\]

\newpage \begin{example}
\protect\hypertarget{exm:unlabeled-div-13}{}\label{exm:unlabeled-div-13}

Consider the following problems.

\begin{enumerate}
\def\labelenumi{\arabic{enumi}.}
\item
  \emph{Express a nominal annual interest rate of 9\% convertible
  half-yearly as a monthly effective interest.}
\item
  \emph{Express a two-monthly effective interest of 3\% as a nominal annual
  interest rate convertible two-monthly.}
\end{enumerate}

\end{example}

\textbf{Solution:}

\begin{enumerate}
\def\labelenumi{\arabic{enumi}.}
\item
  The effective rate \(i\)\% p.a. is \[i = ( 1 + \frac{0.09}{2})^2 - 1.\]
  Hence the monthly effective rate is
  \(j = (1 + i)^{1/12} - 1 = ( 1 + \frac{0.09}{2})^{2/12} - 1 = 0.007363\).
\item
  A nominal annual interest rate convertible two-monthly is
  \(6 \cdot 3\% = 18\%\).
\end{enumerate}

\newpage \begin{example}
\protect\hypertarget{exm:unlabeled-div-14}{}\label{exm:unlabeled-div-14}

\emph{Express each of the following effective rates per annum as a nominal
rate, and vice versa.}

\begin{longtable}[]{@{}ll@{}}
\toprule
\textbf{\emph{Effective Rate}} & \textbf{\emph{Nominal Rate}} \\
\midrule
\endhead
\(i\) = 0.04 & \(i^{(4)} = 0.039412\) \\
\(i\) = 0.10 & \(i^{(12)} = 0.095690\) \\
\(i\) = 0.06152 & \(i^{(2)} = 0.06\) \\
\(i\) = 0.126825 & \(i^{(12)} = 0.12\) \\
\bottomrule
\end{longtable}

\end{example}

\hypertarget{compounding-over-any-number-of-time-units}{%
\newpage \subsection{Compounding over any number of time units}\label{compounding-over-any-number-of-time-units}}

Suppose an amount ฿1 is invested at the rate of \(i\)\% per time unit. At
time \(t\) the accumulation is \((1 + i)^t\).

\newpage \begin{example}
\protect\hypertarget{exm:unlabeled-div-15}{}\label{exm:unlabeled-div-15}

\begin{enumerate}
\def\labelenumi{\arabic{enumi}.}
\item
  \emph{An investor invests ฿4000 at} \(8.5\)\% per quarter effective. What
  will be the deposit have accumulated to after 1 month.
\item
  \emph{An investor invests ฿800 at} \(6\)\% per half-year effective. What
  will be the deposit have accumulated to after 2.6 years.
\end{enumerate}

\end{example}

\textbf{Solution:}

\begin{enumerate}
\def\labelenumi{\arabic{enumi}.}
\item
  The accumulation after 1 month is
  \(4000 \cdot 1.085^{1/3} = 4110.265768.\)
\item
  The accumulation after 2.6 years is
  \(800 \cdot 1.06^{5.2} = 1083.129754.\)
\end{enumerate}

\begin{exercise}
\protect\hypertarget{exr:unlabeled-div-16}{}\label{exr:unlabeled-div-16}

(Excel) Use Excel to create a table showing the accumulated amounts after 1 year under several different compounding frequencies (yearly, quarterly, monthly, daily) for a principal of ฿100 under with nominal rate of \(r = 4\%\) per annum.

Discuss the results obtained. What happens if the compounding is made over infinitely small intervals (i.e.~as \(m \rightarrow \infty\))?

\end{exercise}

\hypertarget{changing-the-time-period-of-the-effective-rates-of-interest}{%
\newpage \subsection{Changing the time period of the effective rates of interest}\label{changing-the-time-period-of-the-effective-rates-of-interest}}

It is often very useful to change the effective rate of interest per
time period to another. For example, if the effective rate of interest
is defined per annum but cashflows occur monthly.

Let \(i\) be the effective rate of interest per \(t_i\) years (which can be
any positive number, for e.g.~\(t_i = 1/2\)). Here \(t_i\) years can be
regarded as one time unit. Let \(j\) be the effective rate of interest per
\(t_j\) years.

\newpage \begin{example}
\protect\hypertarget{exm:unlabeled-div-17}{}\label{exm:unlabeled-div-17}

\emph{Find the condition under which the two effective rates of interest} \(i\)
and \(j\) are equivalent.

\end{example}

\textbf{Solution:} Suppose we invest 1 for one year. Then at the end of the
year under each rate of interest, we will have
\[(1+i)^{1/t_i} \text{ and } (1+j)^{1/t_j}.\] Two rates of interest are
equivalent if the given amount of principal invested for the same length
of time produces the same accumulated value, i.e.
\[(1+i)^{1/t_i} = (1+j)^{1/t_j}.\] Solving the equation for \(j\) yields
\[j = (1+i)^{t_j/t_i} - 1.\]

\newpage \begin{example}
\protect\hypertarget{exm:unlabeled-div-18}{}\label{exm:unlabeled-div-18}

\begin{enumerate}
\def\labelenumi{\arabic{enumi}.}
\item
  \emph{If the effective rate of interest is 6\% per annum, what is the
  effective rate of interest per half-year?}
\item
  \emph{If the effective rate of interest is 12\% per two-years effective,
  what is the effective rate of interest per quarter-year?}
\item
  \emph{If the effective rate of interest is 2\% per month effective, what
  is the effective rate of interest per 1.5-years?}
\end{enumerate}

\end{example}

\textbf{Solution:}

\begin{enumerate}
\def\labelenumi{\arabic{enumi}.}
\item
  \(i = 6\%\) p.a. Then
  \[j = (1.06)^{1/2} -1 = 0.029563 \text{ per half-year}.\]
\item
  \(i = 12\%\) per two-years. Then
  \[j = (1.12)^{1/(2\times4)} -1 = 0.0142669 \text{ per quarter-year}.\]
\item
  \(i = 2\%\) per month. Then
  \[j = (1.02)^{1.5/(1/12)} -1 = 0.428246 \text{ per 1.5-years}.\]
\end{enumerate}

\hypertarget{non-constant-interest-rates}{%
\newpage \subsection{Non-constant interest rates}\label{non-constant-interest-rates}}

The effective rate may not be the same during every time period. We
shall assume that the rates in every future time periods are known in
advance.

\newpage \begin{example}
\protect\hypertarget{exm:unlabeled-div-19}{}\label{exm:unlabeled-div-19}

\emph{The effective rate of interest per annum was 4\% during 2015, 4.5\%
during 2016 and 5\% during 2017. Calculate the accumulation of ฿200
invested on}

\begin{enumerate}
\def\labelenumi{\arabic{enumi}.}
\item
  \emph{01/01/2015 for 3 years}
\item
  \emph{01/07/2015 for 2 years}
\item
  \emph{01/04/2016 for 1.5 years}
\end{enumerate}

\end{example}

\textbf{Solution:}

\begin{enumerate}
\def\labelenumi{\arabic{enumi}.}
\item
  Accumulating the ฿200 for the first year at the rate of 4\% p.a.
  gives \[200 \cdot 1.04.\] The accumulated value was then invested at
  the rate of 4.5\% p.a. for another year, and its value at after 2
  years was \[200 \cdot 1.04 \cdot 1.045.\] At the rate of 5\% in the
  final year, the value after 3 years was
  \[200 \cdot 1.04 \cdot 1.045 \cdot 1.05 = 228.228.\]
\item
  The accumulation is
  \[200 \cdot 1.04^{1/2} \cdot 1.045 \cdot 1.05^{1/2} = 218.4025.\]
\item
  The accumulation is
  \[200 \cdot 1.045^{9/12} \cdot 1.05^{3/4} = 214.416986.\]
\end{enumerate}

\hypertarget{accumulation-factors}{%
\newpage \subsection{Accumulation factors}\label{accumulation-factors}}

Let \(i\) be the effective rate of interest per one time unit and \(s < t\).
We define

\begin{itemize}
\item
  the accumulation factor per one time unit \[A(0,1) = (1 + i).\]
\item
  the accumulation factor per \(t\) time units \[A(0,t) = (1 + i)^t.\]
\item
  the accumulation factor at time \(t\) of 1 unit invested at time \(s\)
  \[A(s,t).\]
\end{itemize}

\newpage \begin{example}
\protect\hypertarget{exm:unlabeled-div-20}{}\label{exm:unlabeled-div-20}

\emph{The effective rate of interest per annum was 6\% during 2015, 8\% during
2016 and 10\% during 2017. Calculate the following accumulation factors.}

\begin{enumerate}
\def\labelenumi{\arabic{enumi}.}
\item
  \(A(01/01/15, 01/01/18),\) i.e.~the accumulation at 01/01/18 of an
  investent of 1 at 01/01/15
\item
  \(A(01/07/15, 01/07/17)\)
\item
  \(A(01/04/16, 01/10/17)\)
\end{enumerate}

\end{example}

\textbf{Solution:}

\begin{enumerate}
\def\labelenumi{\arabic{enumi}.}
\item
  \(A(01/01/15, 01/01/18) = (1.06)(1.08)(1.1) = 1.25928\)
\item
  \(A(01/07/15, 01/07/17) = (1.06)^{1/2}(1.08)(1.1)^{1/2} = 1.166200\)
\item
  \(A(01/04/16, 01/10/17) = (1.08)^{3/4}(1.1)^{3/4} = 1.137922\)
\end{enumerate}

\hypertarget{present-values-and-discount-factors}{%
\newpage \subsection{Present values and discount factors}\label{present-values-and-discount-factors}}

Recall from Example \ref{exm:egpv} that the amount
\(\displaystyle{\frac{10000}{1.04^2}}\) we need to invest now to obtain
฿10000 in two years is called the \emph{present value (PV)} or \emph{discounted
value} of the payments.

We define the discount factor \(v\) per annum, at rate \(i\) p.a. effective
to be the present value of a payment of 1 due in 1 year?s time, i.e.
\[v = \frac{1}{1+i}.\]

\newpage \begin{example}
\protect\hypertarget{exm:unlabeled-div-21}{}\label{exm:unlabeled-div-21}

\emph{Calculate the present of ฿25000 due in 3 years at an effective rate of
interest of 6\% per annum.}

\end{example}

\textbf{Solution:} The present value is
\[25000 \cdot \frac{1}{1.06^3} = 20990.482076.\] It is the discounted
value of 25000 due in 3 years.

\newpage \begin{example}
\protect\hypertarget{exm:unlabeled-div-22}{}\label{exm:unlabeled-div-22}

\emph{How much should we invest now to meet a liability of ฿50000 in 5 years
at an effective rate of interest of 3\% per half-year.}

\end{example}

\textbf{Solution:} The amount we need to invest now to meet the future
liability of 50000 in 5 years is the present value
\[50000 \cdot \frac{1}{1.03^{10}} = 37204.695745.\]

\textbf{Note} It follows that the \(PV\) of ฿1 in \(t\) time units at \(i\)
effective rate of interest per time unit is
\[PV = \frac{1}{(1+i)^t} = v^t.\]

\newpage \begin{example}
\protect\hypertarget{exm:unlabeled-div-23}{}\label{exm:unlabeled-div-23}

\emph{Given the discount factor per year} \(v = 0.9\), calculate

\begin{enumerate}
\def\labelenumi{\arabic{enumi}.}
\item
  \emph{the effective rate of interest per year.}
\item
  \emph{the equivalent discount factor per half-year.}
\end{enumerate}

\end{example}

\textbf{Solution:}

\begin{enumerate}
\def\labelenumi{\arabic{enumi}.}
\item
  From \(\displaystyle{ v= \frac{1}{1+i} = 0.9}\), solving the equation
  for \(i\) gives \[i = \frac{1}{v} - 1 = 0.111111 \text{ per year}.\]
\item
  Let \(j\) be the effective rate of interest per half-year. Then
  \[j = (1+ i)^{1/2} -1 = 0.054093.\] Then, the discount factor per
  half-year is \[v = \frac{1}{1+j} = \frac{1}{1.054093} = 0.948683.\]
\end{enumerate}

Similarly, we define

\begin{itemize}
\item
  the discount factor per one time unit \[V(0,1) = 1/(1 + i).\]
\item
  the discount factor per \(t\) time units \[V(0,t) = 1/(1 + i)^t.\]
\item
  for \(s < t\), the discount factor at time \(s\) of 1 unit receivable at
  time \(t\) \[V(s,t) =  (1 + i)^{s - t}.\]
\end{itemize}

\textbf{Notes}

\begin{enumerate}
\def\labelenumi{\arabic{enumi}.}
\item
  \(V(s,t) = A(s,t)^{-1}\)
\item
  For \(r < s < t\), the following holds:

  \begin{itemize}
  \item
    \(A(r,t) = A(r,s) A(s,t)\)
  \item
    \(V(r,t) = V(r,s) V(s,t)\)
  \end{itemize}
\end{enumerate}




\hypertarget{cashflows-and-annuities}{%
\section{Cashflows and Annuities}\label{cashflows-and-annuities}}

Consider a series of cashflows defined by (see the timeline)

\begin{enumerate}
\def\labelenumi{\arabic{enumi}.}
\item
  the times of payments (cashflows), denoted by \(t_1, t_2, \ldots,\)
  and
\item
  the amount of payments, denoted by \(C_{r}\) (or \(C_{t_r}\)), which
  will be paid at time \(t_r\), for \(r = 1,2, \ldots\). The amounts can
  be positive or negative
\end{enumerate}

The present value at any time \(t\) of this series of cashflow is
\[PV(t) = \sum_{r=1}^\infty C_r (1 + i)^{t - t_r} = \sum_{r=1}^\infty C_r v^{t _r - t}\]
where \(i\) is the effective rate of interest.

The above formula can be obtained by summing these two components:

\begin{itemize}
\item
  for all \(t_r < t\), adding up the accumulations of these individual
  cashflows up to time \(t\), and
\item
  for all \(t_r > t\) , adding up the discounted values of these
  individual cashflows back to time \(t\).
\end{itemize}

\begin{center}\includegraphics{tikz-ex7-1} \end{center}

\textbf{Notes}

\begin{enumerate}
\def\labelenumi{\arabic{enumi}.}
\item
  At a fixed effective rate of interest, the original series of
  cashflows is equivalent to a single payment of amount \(PV(t)\) at
  time \(t\).
\item
  If two different series of cashflows have the same \(PV\) at one time
  at a given effective rate of interest, then they have the same \(PV\)
  at any time at that effective rate of interest.
\end{enumerate}

\newpage \begin{example}
\protect\hypertarget{exm:unlabeled-div-24}{}\label{exm:unlabeled-div-24}

\emph{Let} \(i = 4\%\) effective per time unit. Cashflows are given as follows:

\begin{itemize}
\item
  \(C_1 = 200\) at time \(t_1 = 1\).
\item
  \(C_2 = 300\) at time \(t_2 = 3\).
\item
  \(C_3 = -100\) at time \(t_3 = 5\).
\item
  \(C_4 = -50\) at time \(t_4 = 6\).
\end{itemize}

\emph{Calculate}

\begin{enumerate}
\def\labelenumi{\arabic{enumi}.}
\item
  \emph{the accumulation at time} \(t = 7\).
\item
  \emph{the present value at time} \(t = 0\).
\item
  \emph{the present value at time} \(t = 4\).
\end{enumerate}

\end{example}

\textbf{Solution:}

\begin{center}\includegraphics{tikz-ex8-1} \end{center}

\begin{enumerate}
\def\labelenumi{\arabic{enumi}.}
\item
  The series of cashflows is shown in the following timeline. The
  accumulation at time \(t = 7\) is \[\begin{aligned}
      \sum_{r=1}^4 A(t_r,7) &= 200 \cdot A(1,7) +  300 \cdot A(3,7) -  100 \cdot A(5,7) -  50 \cdot A(6,7) \\
      &= 200 \cdot 1.04^6 + 300 \cdot 1.04^4 - 100 \cdot 1.04^2 - 50 \cdot 1.04 \\
      & = 443.861372\end{aligned}\]
\item
  The present value at time \(t = 0\) can be obtained by discounting the
  accumulation at time \(t = 7\) back to time \(t = 0\), which is
  \[443.861372  \cdot V(0,7) = 443.861372  \cdot \frac{1}{1.04^7} = 337.298163.\]
\item
  The present value at time \(t = 4\) is
  \[443.861372  \cdot V(4,7) = 443.861372  \cdot \frac{1}{1.04^3} = 394.591143.\]
\end{enumerate}

\hypertarget{level-annuities-certain}{%
\subsection{Level Annuities certain}\label{level-annuities-certain}}

An \textbf{annuity} is a series of payments made at equal intervals. There are many practical examples of financial transactions involving annuities, such as.

\begin{itemize}
\item
  a car loan that is repaid in equal monthly instalments
\item
  a pensioner who purchases an annuity from an insurance company upon retirement
\item
  a life insurance policy that is taken out with monthly premiums
\end{itemize}

When certain payments are to be made for a certain period of time, they are called \emph{annuity certain}.

\begin{itemize}
\item
  If the payments are made at the end of each time period, they are paid \emph{in arrears}.
\item
  Otherwise, payments are made at the beginning of each time period,
  they are pain \emph{in advance}.
\item
  An annuity paid in advance is also known as an \emph{annuity due}
\item
  If each payment is for the same amount, this is a \emph{level} annuity.
\end{itemize}

\newpage \begin{example}
\protect\hypertarget{exm:unlabeled-div-25}{}\label{exm:unlabeled-div-25}

\emph{Let} \(i\) be the constant effective rate of interest per time unit. Show
that the accumulated value of a level annuity certain having cashflow of
1 unit at the end of each of the next \(n\) time units is
\[\frac{(1+i)^n -1 }{i}.\] Such accumulated value of the annuity is
denoted by \(s_\angl{n}\) (pronounced ``S.N.'')

\end{example}

\begin{center}\includegraphics{tikz-ex9-1} \end{center}

\textbf{Solution:} Based on the first principles,

\begin{align} 
 s_\angl{n} &= \sum_{r=1}^n C_r \cdot A(t_r,n) \\
    &= (1+i)^{n-1} + (1+i)^{n-2} + \ldots + (1+i) + 1. \label{eq:firstEq} 
\end{align}

Multiplying Eq.\eqref{eq:firstEq} through by (1+i) gives \begin{equation}
    (1+i) \cdot s_\angl{n}  = (1+i)^{n} + (1+i)^{n-1} + \ldots + (1+i)^2 + (1+i). 
\end{equation} Subtracting the two equations results in
\[\begin{aligned}
    i \cdot s_\angl{n} &= (1+i)^{n} - 1\\
        s_\angl{n} &= \frac{(1+i)^{n} - 1}{i}.\end{aligned}\]

\newpage \begin{example}
\protect\hypertarget{exm:unlabeled-div-26}{}\label{exm:unlabeled-div-26}

\emph{Let} \(i\) be the constant effective rate of interest per time unit. Show
that the present value at time 0 of a level annuity certain, denoted by
\(a_\angl{n}\) (pronounced ``A.N.'') , having cashflow of 1 unit at the end
of each of the next \(n\) time units is
\[a_\angl{n} = \frac{1 - v^n }{i}.\]

\end{example}

\textbf{Solution:} Taking the accumulated value at time \(n\) and discounting
back to time 0 gives \[\begin{aligned}
    a_\angl{n} &= s_\angl{n} \cdot v^n \\
            &= \frac{(1+i)^{n} - 1}{i} \cdot v^n \\
            &=  \frac{1 - v^n }{i}.\end{aligned}\]

\newpage \begin{example}
\protect\hypertarget{exm:unlabeled-div-27}{}\label{exm:unlabeled-div-27}

\emph{Given the effective rate of interest of} \(8\%\) p.a., calculate

\begin{enumerate}
\def\labelenumi{\arabic{enumi}.}
\item
  \emph{the accumulation at 12 years of ฿500 payable yearly in arrears for
  the next 12 years.}
\item
  \emph{the present value now of ฿2,000 payable yearly in arrears for the
  next 6 years.}
\item
  \emph{the present value now of ฿1,000 payable half-yearly in arrears for
  the next 12.5 years.}
\end{enumerate}

\end{example}

\textbf{Solution:}

\begin{enumerate}
\def\labelenumi{\arabic{enumi}.}
\tightlist
\item
  The timeline of this transaction is shown in the figure below.
\end{enumerate}

\begin{center}\includegraphics{tikz-ex10-1} \end{center}

\begin{verbatim}
The accumulation of the payments is
$$500 \cdot s_\angl{12}  = 500 \cdot \frac{1.08^{12} - 1}{0.08} = 9488.563230.$$
\end{verbatim}

\begin{enumerate}
\def\labelenumi{\arabic{enumi}.}
\setcounter{enumi}{1}
\item
  The present value of the payments is
  \[2000\cdot a_\angl{6}  = 2000 \cdot \frac{1 - 1.08^{-6} }{0.08 } = 9245.759328.\]
\item
  An interest rate of 8\% p.a. is equivalent to an effective
  half-yearly interest rate, denoted by \(j\), of
  \[j = 1.08^{1/2} -1 = 0.039230.\] There are 25 payments of 1000
  each, starting in six months' time.

  Working in terms of half year, the present value of the payment is
  \[1000 \cdot a^j_\angl{25} = 1000 \cdot \frac{1 - 1.039230^{-25} }{0.039230 } = 15750.003911.\]
\end{enumerate}

\newpage \hypertarget{level-annuities-due}{%
\subsection{Level Annuities Due}\label{level-annuities-due}}

An \emph{annuity-due} is an annuity where the payments made at the start of
each time period (instead of a the end), i.e.~the payments are paid \emph{in
advance}.

In order to calculate the present value or accumulation of an annuity
due, we first introduce the concept of the rate of discount.

\hypertarget{the-rate-of-discount}{%
\subsubsection*{The rate of discount}\label{the-rate-of-discount}}
\addcontentsline{toc}{subsubsection}{The rate of discount}

As opposed to the interest rate where the accumulation of initial
investment can be obtained by multiplying it by the accumulation factor
\((1+i)^n\), we can obtain the discounted value of payment by using
discount rates.

Suppose an amount of ฿1 is due after 1 year with an effective rate of
\(i \%\) p.a. (see the timeline below). What is the amount of money
required to invested now to accumulate to 1?

\begin{center}\includegraphics{tikz-ex12-1} \end{center}

The amount of money required now to accumulate to ฿1 in one year is
\[v =  \frac{1}{1+i}.\] Note that \[\frac{1}{1+i} = 1 - \frac{i}{1+i}.\]
We define the effective rate of discount \(d\) per annum
as\[d = \frac{i}{1+i}.\] It follows that
\[v = \frac{1}{1+i} = 1 - \frac{i}{1+i} =  1 - d\] represents the
discount of ฿1 for 1 year using the effective rate of interest of \(i \%\)
p.a.

Similarly, suppose an amount of ฿1 is due after \(n\) year with an
effective rate of \(i \%\) p.a. The amount of money required to invested
now to accumulate to 1 in \(n\) year is \[\frac{1}{(1+i)^n} = (1-d)^n.\]
See the timeline below for illustration.

\begin{center}\includegraphics{tikz-ex14-1} \end{center}

\newpage \begin{example}
\protect\hypertarget{exm:unlabeled-div-28}{}\label{exm:unlabeled-div-28}

\emph{Discount ฿2,000 for 3 years using the effective rate of discount of 5\%
per annum.}

\end{example}

\textbf{Solution:} After 1 year the discount will be \(0.05 \cdot 2000 = 100,\)
and the discounted value of the payment will be
\[2000 \cdot (1 - d) = 2000 \cdot (1 - 0.05) = 1900 .\] Similarly, after
2 years, the discounted value will be
\[2000 \cdot (1 - d)^2 = 2000 \cdot (1 - 0.05)^2 = 1805 .\] After 3
years, the discounted value of the payment will be
\[2000 \cdot (1 - d)^3 = 2000 \cdot (1 - 0.05)^3 = 1714.75 .\]

\newpage \begin{example}
\protect\hypertarget{exm:unlabeled-div-29}{}\label{exm:unlabeled-div-29}

\emph{The effective rate of discount} \(d\) per time unit can be regarded as
the interest paid in advance at time 0, which is equivalent to the
effective rate of interest \(i\) payable in arrears.

\end{example}

\textbf{Solution:} To show this, suppose that the bank added interest of \(x\)
to an account of an amount of 1 unit at the start of the period. Assume
that the interest amount of \(x\) can be withdrawn and invested in another
bank that earn the rate of interest \(i\%\) effective per time unit. The
principle of 1 unit is still in the first bank.

\begin{center}\includegraphics{tikz-ex15-1} \end{center}

At the end of the year, we have

\begin{itemize}
\item
  the principle of 1 unit in the first bank, and
\item
  the interest paid in advance which accumulates to \(x(1+i)\) in the
  second bank.
\end{itemize}

For this to be equivalent to the interest paid in arrears, we can find
\(x\) which solves \[\begin{aligned}
     1 + x(1+i) &= 1 + i,\\
     x &= \frac{i}{1+i} = \frac{1+i}{1+i} - \frac{1}{1+i}  = 1-v = d.\end{aligned}\]
Therefore, the effective rate of discount \(d\) per time unit can be
regarded as the interest paid in advance at time 0, which is equivalent
to the effective rate of interest \(i\) payable in arrears.

\newpage \begin{example}
\protect\hypertarget{exm:unlabeled-div-30}{}\label{exm:unlabeled-div-30}

\emph{Let} \(i\) be the constant effective rate of interest per time unit. Show
that the accumulated value of a level annuity due, denoted by
\(\ddot{s}_\angl{n}\) (pronounced ``S-due N'', having cashflow of 1 unit at
the start of each of the next \(n\) time units is
\[\ddot{s}_\angl{n} = \frac{(1+i)^n -1 }{d}.\]

\end{example}

\begin{center}\includegraphics{tikz-ex16-1} \end{center}

\textbf{Solution:} Using the previous results, it follows that
\[\begin{aligned}
 \ddot{s}_\angl{n} &= (1+i)^n + (1+i)^{n-1} + \cdots + (1+i)^2 + (1+i) \\
            &= (1+i) \cdot \left[(1+i)^{n-1} + \cdots + (1+i)^1 + 1\right] \\
            &= (1+i) \cdot {s}_\angl{n} \\
            &= (1+i) \cdot \frac{(1+i)^n -1 }{i}\\
            &=  \frac{(1+i)^n -1 }{i/(1+i)}\\
            &=  \frac{(1+i)^n -1 }{d}.\end{aligned}\]

\newpage \begin{example}
\protect\hypertarget{exm:unlabeled-div-31}{}\label{exm:unlabeled-div-31}

\emph{Let} \(i\) be the constant effective rate of interest per time unit. Show
that the present value at time 0 of a level annuity due having cashflow
of 1 unit at the start of each of the next \(n\) time units is
\[\ddot{a}_\angl{n} =  \frac{1 - v^n }{d}.\]

\end{example}

\textbf{Solution:} The present values of the payments can be obtained by
discounting \(\ddot{a}_\angl{n}\) back to time 0, i.e.~\[\begin{aligned}
 \ddot{a}_\angl{n} &= v^n  \cdot  \ddot{s}_\angl{n} \\
            &= v^n  \frac{(1+i)^n -1 }{d} \\
            &=\frac{1 - v^n}{d}.\end{aligned}\]

\newpage \begin{example}
\protect\hypertarget{exm:unlabeled-div-32}{}\label{exm:unlabeled-div-32}

\emph{Given the effective rate of interest of} \(8\%\) p.a., calculate

\begin{enumerate}
\def\labelenumi{\arabic{enumi}.}
\item
  \emph{the accumulation at 12 years of ฿500 payable yearly in advance for
  the next 12 years.}
\item
  \emph{the present value now of ฿2,000 payable yearly in advance for the
  next 6 years.}
\item
  \emph{the present value now of ฿1,000 payable half-yearly in advance for
  the next 12.5 years.}
\end{enumerate}

\end{example}

\textbf{Solution:}

\begin{enumerate}
\def\labelenumi{\arabic{enumi}.}
\item
  The accumulation of the annuity-due of 12 years is
  \[500 \cdot \ddot{s}_\angl{12} = 500 \cdot \frac{1.08^{12} -1}{0.08/1.08} = 10247.648289.\]
\item
  The present value of the annuity-due of 6 years is
  \[2000 \cdot \ddot{a}_\angl{6} = 2000 \cdot \frac{1- 1.08^{-6}}{0.08/1.08} = 9985.420074.\]
\item
  An interest rate of 8\% p.a. is equivalent to an effective
  half-yearly interest rate, denoted by \(j\), of
  \[j = 1.08^{1/2} -1 = 0.039230.\] There are 25 payments of 1000
  each, starting in six months' time.

  Working in terms of half year, the present value of the payment is
  \[1000 \cdot \ddot{a}^j_\angl{25} = 1000 \cdot \frac{1 - 1.039230^{-25} }{0.039230/1.039230 } = 16367.876564.\]
\end{enumerate}

\newpage \hypertarget{deferred-annuities}{%
\subsection{Deferred annuities}\label{deferred-annuities}}

An annuity whose first payment is made during the first time period
(either in arrears or in advance) is called \emph{immediate annuity}.
Otherwise, the annuity is known as \emph{deferred} annuity, i.e.~the first
payment starts some time in the future.

To calculate the present value of the annuity of a series of \(n\)
payments deferred for \(m\) time units (the first payment is due at time
\(m+1\)), denoted by \(_{m \textbar}a_\angl{n}\), we first calculate the
present value at the end of the deferred period and then discount back
to the start of the period.

\begin{center}\includegraphics{tikz-ex17-1} \end{center}

\[\begin{aligned}
    _{m \textbar} a_\angl{n}  &= v^{m+1} + v^{m+2} + \cdots +v^{m+n}  \\
    &= v^m  \left( v + v^2 + \cdots + v^n  \right) \\
    &= v^m \cdot  a_\angl{n}.\end{aligned}\]

\newpage \begin{example}
\protect\hypertarget{exm:unlabeled-div-33}{}\label{exm:unlabeled-div-33}

\emph{Calculate the present value at time 0 of an annuity of 1 p.a. in
arrears for 6 years and deferred for 10 at 6\% effective rate p.a.}

\end{example}

\begin{center}\includegraphics{tikz-ex18-1} \end{center}

This is an annuity with 6 unit payments for which the first payment is
at time 11. Hence the present values of such payments is
\[\begin{aligned}
    _{10 \textbar}a_\angl{6}
  &= v^{11} + v^{12} + \cdots +v^{16}  \\
    &= v^{10}  \left( v + v^2 + \cdots + v^6  \right) \\
    &= v^{10} \cdot  a_\angl{6} \\
    &= \left(\frac{1}{1.06}\right)^{10} \cdot \left( \frac{1- 1.06^{-6}}{0.06} \right)
    &= 2.745808.\end{aligned}\]

\newpage \begin{example}
\protect\hypertarget{exm:unlabeled-div-34}{}\label{exm:unlabeled-div-34}

\emph{Give the reason or show that the present value of a series of} \((n+m)\)
payments of one unit payable at the end of each time period is equal to
the sum of

\begin{enumerate}
\def\labelenumi{\arabic{enumi}.}
\item
  \emph{present value of} \(m\) payments of one units payable at the end of
  each time period (denoted by \(a_\angl{m}\)) and
\item
  \emph{present value of} \(n\) payments of one units payable at the end of
  each time period deferred for \(m\) years (denoted by
  \(_{m \textbar}a_\angl{n}\)).
\end{enumerate}

\end{example}

\textbf{Solution:} The present value of a series of (m+n) payments is
\[\begin{aligned}
a_\angl{m+n} &=  \left( v + v^2 + \cdots  v^m\right) + \left( v^{m+1} + v^{m+2} + \cdots +v^{m+n} \right) \\
&= a_\angl{m} + _{m \textbar}a_\angl{n}.\end{aligned}\] It follows that
\(_{m \textbar}a_\angl{n} = a_\angl{m+n} - a_\angl{m}.\)

\newpage \hypertarget{increasing-annuities}{%
\subsection{Increasing annuities}\label{increasing-annuities}}

An annuity in which the \(i\)th payment of the amount \(i\) is made at time
\(t_i = i\) is called an \emph{(simple) increasing} annuity. The present and
accumulated value of this annuity can be obtained from the first
principles. For example, the present value of the increasing annuity can
be evaluated by \[\sum_{i=1}^n X_i v^{t_i} = \sum_{i=1}^n i v^{t_i},\]
where the \(i\)th payment of amount \(X_i = i\) at time \(t_i = i\).

\newpage \begin{example}
\protect\hypertarget{exm:unlabeled-div-35}{}\label{exm:unlabeled-div-35}

\emph{Derive the formula for the present value of a simple increasing annuity
payable yearly in arrears with the effective rate} \(i\%\) p.a. for \(n\)
years.

\end{example}

\textbf{Solution:} The cashflows of the simple increasing annuity payable
yearly in arrears is illustrated below. The present value of payments of
1 at time 1, 2 at time 2, \(\ldots, n\) at time \(n\) denoted by
\((Ia)_\angl{n}\) is given by
\[(Ia)^{i}_\angl{n}   = \frac{\ddot{a}^{i}_{\actuarialangle{n}} - nv^n}{i}.\]

\begin{center}\includegraphics{tikz-ex19-1} \end{center}

\textbf{Notes}

\begin{enumerate}
\def\labelenumi{\arabic{enumi}.}
\item
  An increasing annuity but with payments in advance is given by
  \[(I\ddot{a})^{i}_\angl{n}   = \frac{\ddot{a}^{i}_{\actuarialangle{n}} - nv^n}{d}.\]
\item
  The formulas for the accumulated values are
  \[(Is)^{i}_\angl{n} = \frac{\ddot{s}^{i}_{\actuarialangle{n}} - n}{i} \quad\text{(in arrears)}\]
  \[(I\ddot{s})^{i}_\angl{n} = \frac{\ddot{s}^{i}_{\actuarialangle{n}} - n}{d} \quad\text{(in advance)}\]
\end{enumerate}

\newpage \hypertarget{compound-increasing-annuities}{%
\subsection{Compound increasing annuities}\label{compound-increasing-annuities}}

The following example considers the value of compound increasing
annuities where the payments increase by a constant factor each time.

\newpage \begin{example}
\protect\hypertarget{exm:unlabeled-div-36}{}\label{exm:unlabeled-div-36}

\emph{Assume that the effective rate of interest is 6\% p.a. Calculate the
present value as at 1 January 2010 of an annuity payable annually in
arrears for 8 years. The first payment is ฿10 and subsequent payments
increase by 2\% per annum compound.}

\end{example}

\textbf{Solution:}

\begin{center}\includegraphics{tikz-ex20-1} \end{center}

At 1/1/2010, the present value of the payment is given by
\[\begin{aligned}
    PV &= 10 \cdot \frac{1}{1.06} + 10 \cdot \frac{1.02}{(1.06)^2} + \cdots +  10 \cdot \frac{(1.02)^7}{(1.06)^8} \\
        &= \frac{10}{1.02} \left(  \frac{1.02}{1.06} +  \left(  \frac{1.02}{1.06}  \right)^2 + \cdots +
    \left(  \frac{1.02}{1.06}  \right)^8  \right)\end{aligned}\] The
above equation can be arranged so that the annuity formula can be
applied. We can define \(j\) such that \(1 + j = 1.06/1.02\), and hence,
\[\begin{aligned}
    PV &= \frac{10}{1.02} \left(  \frac{1}{1 + j} +  \left(  \frac{1}{1+j}  \right)^2 + \cdots +
    \left(  \frac{1}{1+j}  \right)^8  \right)  \\
    &=   \frac{10}{1.02}  a_\angl{8} \quad  \text{ at } j\% \\
    &=  \frac{10}{1.02} \left(   \frac{1 - \left(\frac{1.02}{1.06}\right)^8   }{\left(   \frac{1.06}{1.02}   - 1 \right)}  \right) \\
    &= 66.2216\end{aligned}\]

\newpage \hypertarget{annuities-payable-more-than-once-per-time-unit}{%
\subsection{Annuities payable more than once per time unit}\label{annuities-payable-more-than-once-per-time-unit}}

Consider the value of an annuity payable in arrears \(m\) times per time
unit at an effective rate of interest \(i\) per time unit. The annuity is
still payable for \(n\) time units and a total amount of 1 unit per time
unit. The present and accumulated values of the corresponding annuity
are denoted by \(a^{(m)i}_\angl{n}\) and \(s^{(m)i}_\angl{n}\),
respectively.

To calculate either the present or accumulation value of this annuity,
we can simply apply the first principles by using the effective rate of
interest per \(1/m\) time unit. In particular, we have
\[a^{(m)i}_\angl{n}   = \frac{1}{m}  a^j_\angl{n\cdot m},\] and
\[s^{(m)i}_\angl{n}   = \frac{1}{m}  s^j_\angl{n\cdot m},\] where \(j\) is
the effective rate per \(1/m\) time unit.

\newpage \begin{example}
\protect\hypertarget{exm:unlabeled-div-37}{}\label{exm:unlabeled-div-37}

\emph{Calculate the accumulation at 1 January 2020 of an annuity of ฿100 per
month, payable in arrears from 1 January 2010 at an effective rate of
interest of 4\% p.a.}

\end{example}

\textbf{Solution:} The annual payment is 1200 and the effective rate per
month equivalent to 4\% p.a. is \(j = (1.04)^{1/12} - 1 = 0.003274\) per
month. Hence,
\[1200 s^{(12)4\%}_\angl{10}   = 100  s^j_\angl{12\cdot 10} = 14669.59.\]



\newpage \hypertarget{principle-of-equivalence-yields-and-equation-of-value}{%
\section{Principle of Equivalence, Yields and Equation of Value}\label{principle-of-equivalence-yields-and-equation-of-value}}

The principle of equivalence is used to compare two different cashflows
whether one is worth more than the other.

Consider two sequences of cashflows

\begin{itemize}
\item
  \(C_1, C_2, \ldots\) with payments at times \(t_1, t_2, \ldots\) and
\item
  \(D_1, D_2, \ldots\) with payments at times \(s_1, s_2, \ldots\).
\end{itemize}

Assume that the interest rates are given and apply to both of them. The
two sequences of cashflows are said to be \textbf{equivalent} (or equal in
value) if their values at any time \(t\) are the same, i.e.~there exists
\(t \in \mathbb{R}\) such that \[PV^C(t)  = PV^D(t).\]

\textbf{Notes}

\begin{enumerate}
\def\labelenumi{\arabic{enumi}.}
\item
  If two sequences of cashflows have the same value at time \(s\), then
  they have the same value at any time \(t\) since

  \begin{itemize}
  \item
    for \(t \le s\),

    \(\text{(Value at time t)} = \text{(Value at time s)} \times V(t,s),\)
  \item
    for \(t \ge s\)

    \(\text{(Value at time t)} = \text{(Value at time s)} \times A(s,t).\)
  \end{itemize}
\item
  The two sequences of cashflows are \textbf{indifferent} if their present
  values are the same.
\item
  The principle of equivalent can be applied for \textbf{pricing a financial
  security}, for example, a price \(P\) which will be paid by the
  investor in return for a series of future cashflows.
\end{enumerate}

\newpage \begin{example}
\emph{Calculate the maximum price an investor wish to pay in return for an
investment that will pay ฿500 at the end of each of the next 15 months
given that the interest rate is 0.2\% per month.}
\end{example}

The present value of these payments of 500 at the end of the next 15
months is
\[PV(0) = 500 a^{0.002}_{\angl{15}} =  500 \cdot \left(  \frac{1 - (1.002)^{-15}}{0.002}  \right)   = 7381.35.\]
Therefore, the investor would be willing to pay a maximum of 7381.35.

\newpage \begin{example}

\emph{Determine whether the following series of cashflows are equivalent
given that an interest rate is 6\% per annum effective.}

\begin{enumerate}
\def\labelenumi{\arabic{enumi}.}
\item
  \emph{One single payment of amount 6,691.127888 at year 5.}
\item
  \emph{a level annuity of 300 payable yearly in arrears for the next 5
  years plus a lump sum of 5,000.}
\item
  \emph{a level annuity of 1,186.982002 payable yearly in arrears for the
  next 5 years.}
\end{enumerate}

\end{example}

\textbf{Solution:}

\begin{enumerate}
\def\labelenumi{\arabic{enumi}.}
\item
  The present value is \(6,691.127888 \times (1.06)^{-5} = 5000\).
\item
  The present value is
  \[300 a^{0.06}_{\angl{5}}   + 5000 \times (1.06)^{-5}  = 5000.\]
\item
  The present value is \[1186.982002 a^{0.06}_{\angl{5}}     = 5000.\]
\end{enumerate}

Therefore, the three series of cashflows are \textbf{indifferent}.

\newpage \hypertarget{equation-of-value-and-yields}{%
\subsection{Equation of value and yields}\label{equation-of-value-and-yields}}

Consider a transaction from an investment that offers

\begin{itemize}
\item
  to pay an investor of amounts (i.e.~money received)
  \(B_1, B_2, \ldots, B_n\) at time \(t_1, t_2, \ldots ,t_n\)
\item
  in return for outlays (i.e.~money paid out) of amounts
  \(A_1, A_2, \ldots, A_n\) at these times, respectively.
\end{itemize}

Only one of \(A_i\) and \(B_i\) will be non-zero in general.

\textbf{An equation of value} equates the present value of money received to
the present value of money paid out, which can be written as
\[\sum_{i=1}^n A_i v^i = \sum_{i=1}^n B_i v^i.\] The equation of value
can also be written in terms of the \textbf{net cashflow} at time \(t_i\), i.e.
\(C_t = B_t - A_t\), \[PV_i(0) = \sum_{i=1}^n C_i v^i =0.\]

Equations of value are used throughout actuarial work. Some examples are
as follows:

\begin{itemize}
\item
  The \textbf{fair price} to pay for an investment such as a fixed interest
  security or an equity (ie, \(PV\) outgo) equals the present value of
  the proceeds from the investment, discounted at the rate of interest
  required by the investor.
\item
  The \textbf{premium} for an insurance policy is calculated by equating
  the present value of the expected amounts received in premiums to
  the present value of the expected benefits and other outgo.
\end{itemize}

We shall be concerned mainly with the question:

At what rate of interest does the series of amounts paid out have the
same value as the series of amounts received? The corresponding rate of
interest is called the \textbf{yield of the cashflows} (or \textbf{internal rate of
return, money-weighted rate of return}).

\textbf{Notes}

\begin{enumerate}
\def\labelenumi{\arabic{enumi}.}
\item
  Equations of values may have no roots, a unique root or multiple
  roots.
\item
  In most practice situations, there is a unique positive real root.
\end{enumerate}

\newpage \begin{example}
\emph{An investor pays ฿1,000 in order to receive ฿600 back in 2 years and
฿800 back in 4 years. Calculate the annual effective rate of interest
earned on this investment (or the yield on the investment).}
\end{example}

\textbf{Solution:} The yield of the investment \(i\%\) satisfies the equation
of value \[PV_i(0) =  -1000 + 600(1+i)^{-2} + 800(1+i)^{-4} = 0.\] To
solve the equation for \(i\), we define \(z = (1+i)^{-2}\), resulting in
\[8z^2 + 6z - 10 = 0.\] Therefore \(z = 0.804248\) and \(i = 0.115078.\)

\newpage \begin{example}
\protect\hypertarget{exm:exampleYield}{}\label{exm:exampleYield}\emph{An investor pays ฿1,000 in order to receive ฿300 back at the end of the
first 2 years and ฿400 back at the end of the third, forth and fifth
year. Calculate the annual effective rate of interest earned on this
investment (or the yield on the investment).}
\end{example}

\textbf{Solution:} The yield of the investment \(i\%\) p.a. satisfies the
equation of value
\[PV_i(0) =  -1000 + \frac{300}{(1+i)} + \frac{300}{(1+i)^{2}} + \frac{400}{(1+i)^{3}} +  \frac{400}{(1+i)^{4}} + \frac{400}{(1+i)^{5}}.\]
In our next section, we will learn how to approximate the yield of the
above equation.

\newpage \hypertarget{the-method-to-estimate-the-yield}{%
\subsection{The method to estimate the yield}\label{the-method-to-estimate-the-yield}}

By using linear interpolation, the yield can be estimated as follows.
Let \(P_1\) and \(P_2\) be the present values calculated at interest rates
\(i_1\) and \(i_2\), respectively. Then the interest rate corresponding to a
present value of \(P\) can be approximated by
\[i \approx i_1 + (i_2 - i_1) \frac{P - P_1}{P_2 - P_1}.\] In order to
apply this method to calculate the yield \(i\), we simply set \(P = 0\), and
hence \[i \approx i_1 + (i_2 - i_1) \frac{ - P_1}{P_2 - P_1}.\]

From the figure above, the yield \(i\) can be approximated by \(i^*\), which
is the \(x\)-intercept of the straight line joining the points \((i_1,P_1)\)
and \((i_2,P_2)\). From
\[\frac{i^* -i_1}{i_2 - i_1} = \frac{P_{i^*} - P_1}{P_2 - P_1},\] we
have \(P_{i^*} = 0\) and
\[i \approx i^* =  i_1 + (i_2 - i_1) \frac{ - P_1}{P_2 - P_1}.\] Note
that one can get a good approximation by taking values that are either
side of the true value and about 1\% apart.

\newpage \begin{example}
\emph{Approximate the yield of the transaction in Example}
\ref{exm:exampleYield}\emph{.}
\end{example}

\textbf{Solution:} Here, When \(i_1 = 0.21\), \(P_1 = PV_{0.21}(0) = 19.448\) and
when \(i_1 = 0.22\), \(P_2 = PV_{0.22}(0) = -3.698.\) The yield is
approximately equal to \[\begin{aligned}
 i &\approx 0.21 - (0.22 - 0.21) \left(  \frac{19.448}{-3.698 - 19.448} \right) \\
   &= 0.218402 \text{ p.a. effective.}\end{aligned}\]

\newpage \hypertarget{loan-schedules}{%
\section{Loan schedules}\label{loan-schedules}}

In this section, we describe how a loan may be repaid. A schedule of
repayment together with the interest and capital components of an
annuity payment will be discussed.

Suppose that a lender lends an individual of amount \(L\) for \(n\) years
with an effective rate of interest \(i\) per annum. We say that the
\textbf{term} of the loan is \(n\) years with the loan \textbf{amount} of \(L\). How
could we repay the loan?

\newpage \hypertarget{repay-as-late-as-possible}{%
\subsection*{Repay as late as possible:}\label{repay-as-late-as-possible}}
\addcontentsline{toc}{subsection}{Repay as late as possible:}

After \(n\) year, the borrower repays the entire loan and all interest
that accrued over the period. The total amount to be repaid is equal to

\newpage \hypertarget{repay-interest-only-during-the-term-and-repay-the-capital-at-the-end-of-the-term}{%
\subsection*{Repay interest only during the term and repay the capital at the end of the term:}\label{repay-interest-only-during-the-term-and-repay-the-capital-at-the-end-of-the-term}}
\addcontentsline{toc}{subsection}{Repay interest only during the term and repay the capital at the end of the term:}

These types of loan where the borrower is a government or a company are
\textbf{bonds} or \textbf{fixed interest securities}.

\newpage \hypertarget{repay-loan-by-regular-instalments-of-interest-and-capital-throughout-term-of-loan}{%
\subsection*{Repay loan by regular instalments of interest and capital throughout term of loan:}\label{repay-loan-by-regular-instalments-of-interest-and-capital-throughout-term-of-loan}}
\addcontentsline{toc}{subsection}{Repay loan by regular instalments of interest and capital throughout term of loan:}

Each repayment must pay first for interest due and the remainder is used
to repay some of the capital outstanding.

\newpage \begin{example}
\emph{You borrow ฿5,000 for a term of 3 years at a fixed interest rate of 10\%
pa. The loan is to be repaid by 3 level annual repayments of ฿2,010.57
at the end of each year. Calculate the interest content, capital content
from each repayment and capital outstanding after such repayment.}
\end{example}

\textbf{Note} The loan payments can be expressed in the form of a \textbf{Loan
Schedule} as follows:

\begin{longtable}[]{@{}
  >{\centering\arraybackslash}p{(\columnwidth - 8\tabcolsep) * \real{0.0845}}
  >{\centering\arraybackslash}p{(\columnwidth - 8\tabcolsep) * \real{0.1549}}
  >{\centering\arraybackslash}p{(\columnwidth - 8\tabcolsep) * \real{0.2254}}
  >{\centering\arraybackslash}p{(\columnwidth - 8\tabcolsep) * \real{0.2394}}
  >{\centering\arraybackslash}p{(\columnwidth - 8\tabcolsep) * \real{0.2958}}@{}}
\toprule()
\begin{minipage}[b]{\linewidth}\centering
Time
\end{minipage} & \begin{minipage}[b]{\linewidth}\centering
Repayment
\end{minipage} & \begin{minipage}[b]{\linewidth}\centering
Intest content
\end{minipage} & \begin{minipage}[b]{\linewidth}\centering
Capital content
\end{minipage} & \begin{minipage}[b]{\linewidth}\centering
Capital outstanding
\end{minipage} \\
\midrule()
\endhead
0 & & & & 5000 \\
1 & 2010.57 & 500 & 1510.57 & 3489.43 \\
2 & 2010.57 & 348,943 & 1661.627 & 1827.80 \\
3 & 2010.57 & 182.780 & 1827.79 & 0.01 \\
\bottomrule()
\end{longtable}

\newpage \hypertarget{the-loan-schedule}{%
\subsection{The loan schedule}\label{the-loan-schedule}}

A more general form of loan payments can be expressed as follows: Let

\begin{itemize}
\item
  \(L_t\) be the amount of the loan outstanding at time \(t\).
\item
  \(X_t\) be the instalment at time \(t\) (all instalments may not be the
  same amount).
\item
  \(i\) be the effective rate of interest per time unit charged on the
  loan.

  \begin{longtable}[]{@{}
    >{\raggedright\arraybackslash}p{(\columnwidth - 8\tabcolsep) * \real{0.1000}}
    >{\raggedright\arraybackslash}p{(\columnwidth - 8\tabcolsep) * \real{0.1200}}
    >{\raggedright\arraybackslash}p{(\columnwidth - 8\tabcolsep) * \real{0.1800}}
    >{\raggedright\arraybackslash}p{(\columnwidth - 8\tabcolsep) * \real{0.2400}}
    >{\centering\arraybackslash}p{(\columnwidth - 8\tabcolsep) * \real{0.3600}}@{}}
  \toprule()
  \begin{minipage}[b]{\linewidth}\raggedright
  ime Rep
  \end{minipage} & \begin{minipage}[b]{\linewidth}\raggedright
  ayment Int
  \end{minipage} & \begin{minipage}[b]{\linewidth}\raggedright
  est content Ca
  \end{minipage} & \begin{minipage}[b]{\linewidth}\raggedright
  pital content
  \end{minipage} & \begin{minipage}[b]{\linewidth}\centering
  Capital outstanding
  \end{minipage} \\
  \midrule()
  \endhead
  0 & & & & \(L_0\) \\
  1 \$ & X\_1\$ & \(iL_0\) \$ & (X\_1 - iL\_0)\$ & \(L_1 = L_0 - (X_1 - iL_0)\) \\
  2 \$ & X\_2\$ & \(iL_1\) \$ & (X\_2 - iL\_1)\$ & \(L_2 = L_1 - (X_2 - iL_1)\) \\
  \(\vdots\) & & & & \\
  t \$ & X\_t\$ \$ & iL\_\{t-1\}\$ \$(X & \emph{t - iL}\{t-1\})\$ \$L\_ & t = L\_\{t-1\} - (X\_t - iL\_\{t-1\})\$ \\
  \(\vdots\) & & & & \\
  n\$ \$ & X\_n\$ \$ & iL\_\{n-1\}\$ \$(X & \emph{n - iL}\{n-1\})\$ & 0 \\
  \bottomrule()
  \end{longtable}
\end{itemize}

\textbf{Note} The capital outstanding after the \(k\)th payment is
\(X a_{\angl{n-k}}\), which is the present value of future repayments.
This holds even when the repayments and interest rates are not constant.

\newpage \begin{example}

\emph{A loan of ฿20,000 is repayable by equal monthly payments for 4 years,
with interest rate payable at 10\% pa effective.}

\begin{enumerate}
\def\labelenumi{\arabic{enumi}.}
\item
  \emph{Calculate the amount of each monthly payment.}
\item
  \emph{Calculate the interest and capital contents of the 25th repayment.}
\end{enumerate}

\end{example}

\textbf{Solution:}

\begin{enumerate}
\def\labelenumi{\arabic{enumi}.}
\item
  The loan is repaid by level instalments of amount \(X\) payable
  monthly. Working in months, we define \(j\%\) per month effective
  equivalent to 10\% pa effective. We have
  \[j = (1.1)^{(1/12)} - 1 = 0.007974.\] The loan equation followed
  the equation of value is given by
  \[PV_j(0) = 20000 - X a^j_{\angl{48}} = 0\] Solving for \(X\) gives
  \(X =503.12\).
\item
  The capital outstanding after 24th repayment =
  \(L_{24} = X a^j_{\angl{24}} = 10950.23.\) Hence, the interest content
  of the 25th repayment = \(j \cdot L_{24} =87.32.\) The capital content
  of the 25th repayment = \(X =503.12 - 87.32 = 415.8.\)
\end{enumerate}

\newpage \hypertarget{changing-the-term-of-a-loan}{%
\subsection{Changing the term of a loan}\label{changing-the-term-of-a-loan}}

The term of the loan can be changed in the following circumstances:

\begin{itemize}
\item
  extend or shorten the term,
\item
  miss a number of payments,
\item
  repay part of the loan early.
\end{itemize}

The repayment amount will then need to be calculated according to the
condition(s) as given in the change.

\newpage \begin{example}

\emph{A person takes out a loan of ฿100,000 to be repaid by level monthly
instalments in arrears over 7 years where the bank charges an effective
annual rate of interest of 6\%}

\begin{enumerate}
\def\labelenumi{\arabic{enumi}.}
\item
  \emph{Calculate the monthly repayment \textbf{Solution:} Working in months, we
  define} \(j\%\) per month effective equivalent to 6\% pa effective.
  \[j = (1.06)^{(1/12)} - 1 = 0.007974.\] The loan equation followed
  the equation of value is given by
  \[PV_j(0) = 100000 - X a^j_{\angl{84}} = 0\] Solving for \(X\) gives
  \(X =1453.25\).
\item
  \emph{Calculate the new repayment amount if the the term of loan can be
  extended by 1 year, immediately after the 60th repayment has been
  made. \textbf{Solution:} The capital outstanding after 60th repayment =}
  \(L_{60} = X a^j_{\angl{24}} = 32842.48.\) Now the remaining term
  becomes 3 years (or 36 months). The new repayment amount \(X'\)
  satisfies \[PV_j(0) = 32842.48 - X' a^j_{\angl{36}} = 0.\] Solving
  for \(X'\) gives \(X' = 996.77\).
\item
  \emph{Instead of extending the term, the person had requested to miss the
  61st and 62nd repayments. Calculate the remaining installments.
  \textbf{Solution:} After missing the 61st and 62nd repayments, the
  capital outstanding at time 62 =}
  \(L_{60}\cdot (1+j)^2 = 32842.48 (1.004868)^2 = 33162.99.\) Hence, the
  remaining number of payments is 22.
\item
  \emph{Calculate the new repayment amount if the person repaid ฿10,000 at
  the time he made the 60th repayment together with the 60th
  repayment. \textbf{Solution:} The revised capital outstanding after
  repayment of 10000 (the 60th repayment) is}
  \(32842.48 - 10000 = 22842.48.\) The new repayment amount \(X''\)
  satisfies \[PV_j(0) = 22842.48 - X'' a^j_{\angl{24}} = 0.\] Solving
  for \(X''\) gives \(X'' = 1010.76\).
\end{enumerate}

\end{example}

\newpage \hypertarget{changing-the-interest-rate}{%
\subsection{Changing the interest rate}\label{changing-the-interest-rate}}

The interest rates for a loan can vary during the term of the loan. The
reasons for varying rates of interest could be the following:

\begin{enumerate}
\def\labelenumi{\arabic{enumi}.}
\item
  interest rates have been planned to changed during the term, for
  example the borrower would repay less during the beginning of the
  loan, or
\item
  the lender changes the rates of interest to reflect the market
  conditions.
\end{enumerate}

\newpage \begin{example}
\emph{You borrow ฿20,000 for a term of 20 years to be repaid by level annual
instalments. The rate of interest will be 7\% pa effective for the first
10 years and 8\% pa effective thereafter. Calculate the annual
repayment.}
\end{example}

\textbf{Solution:} Let \(X\) be the annual repayment. Using an equation of
value, we have
\[20000 = X a^{7\%}_{\angl{10}} + (1.07)^{-10} X a^{8\%}_{\angl{10}}.\]
Then solving for \(X\) gives \(X = 1916.69\).

\newpage \begin{example}
\emph{You borrow ฿20,000 for a term of 15 years to be repaid by level annual
instalments where the bank charges an effective annual rate of interest
of 6\%. After the 10th repayment has been made, the bank raises the
interest rate to 6.5\% pa effective. Calculate the new repayment amount.}
\end{example}

\textbf{Solution:} The annual repayment \(X\) \textbf{for a term of 15 years} before
the adjustment of interest rate.
\[X = \frac{20000}{a^{6\%}_{\angl{15}}} = 2059.26.\] However, after the
10th repayment has been made, the bank raises the interest rate to 6.5\%
pa effective. Therefore, the capital outstanding after the 10th
repayment = \(L_{10} = X a^{6\%}_{\angl{5}} = 8674.332.\) After the
adjustment of the interest rate, the new repayment amount \(X'\) satisfies
satisfies \[PV_{6.5\%}(0) = 8674.332 - X' a^{6.5\%}_{\angl{5}} = 0.\]
Solving for \(X'\) gives \(X' = 2087.34\).

\end{document}
